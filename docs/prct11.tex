\documentclass{beamer}

\usepackage[utf8]{inputenc}
\usepackage{graphicx}
%\usepackage{color} 

\newtheorem{definicion}{Definición}
\newtheorem{ejemplo}{Ejemplo}

\definecolor{MiVioleta}{RGB}{122,59,122}
\definecolor{MiAzul}{RGB}{0,88,147}
\definecolor{MiGris}{RGB}{56,61,66}
\setbeamercolor*{palette primary}{use=structure, fg=white , bg=MiAzul}
\setbeamercolor*{palette secundary}{use=structure, fg=white , bg=MiVioleta}
\setbeamercolor*{palette tertiary}{use=structure, fg=white , bg=MiGris}

%%%%%%%%%%%%%%%%%%%%%%%%%%%%%%%%%%%%%%%%%%%%%%%%%%%%%%%%%%%%%%%%%%%%%%%%%%%%%%%
\title[Presentación con Beamer]{Numero $\pi$}
\author{Melanie Fumero Padron}
\date[25-04-2014]{25 de abril de 2014}

%%%%%%%%%%%%%%%%%%%%%%%%%%%%%%%%%%%%%%%%%%%%%%%%%%%%%%%%%%%%%%%%%%%%%%%%%%%%%%%

%\usetheme{Madrid}
%\usetheme{Antibes}
%\usetheme{tree}
%\usetheme{classic}

%%%%%%%%%%%%%%%%%%%%%%%%%%%%%%%%%%%%%%%%%%%%%%%%%%%%%%%%%%%%%%%%%%%%%%%%%%%%%%%

\begin{document} % El objetivo de la practica es elaborar un documento sobre el numero PI. 
  
%++++++++++++++++++++++++++++++++++++++++++++++++++++++++++++++++++++++++++++++  
\begin{frame}

  %\includegraphics[width=0.15\textwidth]{img/ullesc}
  \hspace*{7.0cm}
  %\includegraphics[width=0.16\textwidth]{img/fmatesc}
  \titlepage

  \begin{small}
    \begin{center}
     Facultad de Matemáticas \\
     Universidad de La Laguna
    \end{center}
  \end{small}

\end{frame}
%++++++++++++++++++++++++++++++++++++++++++++++++++++++++++++++++++++++++++++++  

%++++++++++++++++++++++++++++++++++++++++++++++++++++++++++++++++++++++++++++++  
\begin{frame}
  \frametitle{Índice}  
  \tableofcontents[pausesections]
\end{frame}
%%%%%%%%%%%%%%%%%%%%%%%%%%%%%%%%%%%%%%%%%%%%%%%%%%%%%%%%%%%%%%%%%%%%%%%%%%%%%%%%%


\section{Historia del numero $\pi$}


\begin{frame}
\frametitle{Historia del numero $\pi$}

La búsqueda del mayor número de decimales del número $\pi$ ha supuesto un esfuerzo constante de numerosos científicos a lo largo de la historia.
Algunas aproximaciones históricas de $\pi$ las mostraremos a continuacion:
\end {frame}
%%%%%%%%%%%%%%%%%%%%%%%%%%%%%%%%%%%%%%%%%%%%%%%%%%%%%%%%%%%%%%%%%%%%%%%%%%%%%
\subsection{Mesopotamia}

\begin {frame}
\frametitle {Mesopotamia}

Algunos matemáticos mesopotámicos empleaban, en el cálculo de segmentos, valores de  igual a 3, alcanzando en algunos casos valores más aproximados, como el de:

\begin{equation}
\pi \approx 3 + \frac{1}{8} = 3,125 
\end{equation}
\end {frame}
%%%%%%%%%%%%%%%%%%%%%%%%%%%%%%%%%%%%%%%%%%%%%%%%%%%%%%%%%%%%%%%%%%%%%%%%%%%%%

\subsection{Matemática islámica}

\begin {frame}
\frametitle{Matematica islamica}

En el siglo IX Al-Jwarizmi, en su Álgebra (Hisab al yabr ua al muqabala), hace notar que el hombre práctico usa 22/7 como valor de $\pi$,
el geómetra usa 3, y el astrónomo 3,1416. En el siglo XV, el matemático persa Ghiyath al-Kashi fue capaz de calcular el valor aproximado de $\pi$ con nueve dígitos,
empleando una base numérica sexagesimal, lo que equivale a una aproximación de 16 dígitos decimales:

\begin{equation}
2 \pi = 6,2831853071795865
\end{equation} 

\end {frame}
%%%%%%%%%%%%%%%%%%%%%%%%%%%%%%%%%%%%%%%%%%%%%%%%%%%%%%%%%%%%%%%%%%%%%%%%%% 

\section{Curiosidades}
 
\begin{frame}

\frametitle{Curiosidades}
\begin{definicion}
En 1983,  Rajan Mahadevan  fue capaz de recitar de memoria 31.811 decimales de $\pi$.
\end{definicion}

\begin{definicion}
No hay ningún 0 entre los 31 primeros dígitos de $\pi$.
\end{definicion}

\begin{definicion}
El 14 de marzo se celebra el dia de $\pi$, y coincide curiosamente con el nacimiento de Einstein~\footnote{Einstein es considerado como el científico más importante del siglo XX}.
\end{definicion}

\end{frame}
%%%%%%%%%%%%%%%%%%%%%%%%%%%%%%%%%%%%%%%%%%%%%%%%%%%%%%%%%%%%%%%%%%%%%%%%%%%%%%  

\section {Mas formulas de $\pi$ }
\begin{frame}

\frametitle{Mas formulas de $\pi$}
\begin{definition}
\begin{equation}
\pi \approx 3 + \frac{1}{8} = 3,125 
\end{equation}
\end{definition}

\begin{definition}
\begin{equation}
\pi \simeq \frac{377}{120} = 3{,}1416 \ldots 
\end{equation}
\end{definition}


\end{frame}
%%%%%%%%%%%%%%%%%%%%%%%%%%%%%%%%%%%%%%%%%%%%%%%%%%%%%%%%%%%%%%%%%%%%%%%%%%%%  

\section{Bibliografía}
%++++++++++++++++++++++++++++++++++++++++++++++++++++++++++++++++++++++++++++++  
\begin{frame}
  \frametitle{Bibliografía}

  \begin{thebibliography}{10}

    \beamertemplatebookbibitems
    \bibitem[Plan Estudios, 2011]{plan}  
    Documento de verificación del grado. 
    (2011) 

    \beamertemplatebookbibitems
    \bibitem[Guía Docente, 2013]{guia}  
    Guía docente. 
    (2013) 
    {\small $http://eguia.ull.es/matematicas/query.php?codigo=299341201$}

    \beamertemplatebookbibitems
    \bibitem[URL: CTAN]{latex} 
    CTAN. {\small $http://www.ctan.org/$}

  \end{thebibliography}
\end{frame}

%++++++++++++++++++++++++++++++++++++++++++++++++++++++++++++++++++++++++++++++  
\end{document}
